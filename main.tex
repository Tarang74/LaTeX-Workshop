%!TEX program = xelatex
% -------------------------------------
% QUT Mathematics Society
% LaTeX Workshop
% -------------------------------------

% -------------------------------------
% Preamble
% -------------------------------------

% In the preamble you define the type of document you are writing,
% load additional packages and set various parameters.

% -------------------------------------
%% Document declaration
% -------------------------------------

% This is the start of every LaTeX document. 

% The document class specifies the overall type of document you are writing. 
\documentclass[11pt, twoside]{article}
% 'article' is most common. 
% 'report' is similar to 'article', but allows for chapters. 
% 'book' is for writing large books. 
% 'beamer' is for making presentations.

% -------------------------------------
%% Packages
% -------------------------------------

% Packages add extra functionality to default LaTeX by adding
% new macros or changing default behaviours.

%% Page layout
\usepackage[a4paper, margin = 2.5cm]{geometry}  % Controls layout of the page, including margins

%% Mathematical environments and symbols
\usepackage{mathtools}                          % Maths equations and aligns (inherits from amsmath)
\usepackage{amsmath}                            % Misc enhancements to math equations
\usepackage{amssymb}                            % More maths symbols
\usepackage{derivative}                         % Derivative symbols

%% Scientific Notation
\usepackage{siunitx}                            % SI unit typesetting
\usepackage[version=4]{mhchem}                  % Chemical equation typesetting

%% Figures
\usepackage{float}                              % Figure positioning
\usepackage{graphicx}                           % Including images
\usepackage{booktabs}                           % For prettier looking tables
\usepackage{subcaption}                         % For using subfigures
\usepackage{array}                              % Column formatting

%% Source code
\usepackage{listings}                           % Source code layout, formatting and styling
\usepackage{color}                              % Needed for colouring source code syntax

%% References
\usepackage[style=ieee]{biblatex}               % Controls bibliography and citations
\usepackage[hidelinks]{hyperref}                % Support for links. Autolinks citations, references, and TOC

%% Lists
\usepackage{enumitem}                           % List environments

%% Unicode support
\usepackage[warnings-off={mathtools-colon, mathtools-overbracket}]{unicode-math}

%% Extra operator example
\DeclareMathOperator{\proj}{proj}

%% Font used by QUT (remove to use original TeX font)
\setmainfont{TeX Gyre Pagella}
\setmathfont{TeX Gyre Pagella Math}

% -------------------------------------
%% Global settings
% -------------------------------------

\lstset{
	language=tex,              % Language for syntax highlighting
	basicstyle=\ttfamily,      % Style for fonts
	numbers=left,              % Line numbers
	numberstyle=\tiny,         % Line number style
	frame=tb,                  % Frame lines on top and bottom
	tabsize=4,                 % Size of tabs
	columns=fixed,             % Fix character columns
	showstringspaces=false,    % Don't underscore spaces
	showtabs=false,            % Don't underscore tabs
	keepspaces,                % Keep spaces as spaces, not whitespace
	commentstyle=\color{red},  % Colour comments as red
	% keywordstyle=\color{blue}, % Colour keywords as blue
	breaklines=true,           % Break lines, so they don't go over the page
}

% -------------------------------------
%% Miscellaneous settings
% -------------------------------------

% Add the references to the bibliography
\addbibresource{sample.bib}

% -------------------------------------
% Title page setup
% -------------------------------------

\title{\LaTeX{} Workshop}
\author{QUT Maths Society}

% Date setup
                    % if nothing is provided, LaTeX uses current date
% \date{6 April 2022} % Specific date
% \date{}             % No date

% -------------------------------------
% End of Preamble
% -------------------------------------

% -------------------------------------
% Document content
% -------------------------------------

% Start of document content
\begin{document}

% -------------------------------------
%% Title page and ToC
% -------------------------------------

% Places the title, author, and date.
\maketitle

% Page number style for table of contents
\pagenumbering{roman}

% Places the table of contents which is generated automatically 
% from the sections
\tableofcontents

% List of figures/tables are also generated automatically 
\listoffigures
\listoftables

% Inserts a new page
\newpage
\pagenumbering{arabic}

% -------------------------------------
%% Document main content
% -------------------------------------

\section{Introduction}
``\LaTeX{} is a high-quality typesetting system; it includes features designed for the production of technical and scientific documentation. \LaTeX{} is the de facto standard for the communication and publication of scientific documents. \LaTeX{} is available as free software''. \parencite{latex_project_latex_2018}

One of the key differences between \LaTeX{} and more common word processors such as MS Word, LibreOffice etc., is the separation of content and presentation. In \LaTeX{}, the author describes the general structure of the document (i.e., section headings, paragraphs, equations, and figures), and the layout and typesetting is handled by \LaTeX{} (or rather the underlying \TeX{} backend).

There are several advantages to this:
\begin{itemize}
    \item The author can focus on the actual content without worrying about layout and presentation
    \item The presentation can be modified without introducing major changes to the document
    \item \LaTeX{}'s standard format allows authors to easily conform to styles provided by external publishers
\end{itemize}
\LaTeX{} is written in plaintext and processed by an external program to generate output files (usually PDFs).
\subsection{Pronunciation and Spelling}
\LaTeX{} is pronounced \emph{lah-tech} or \emph{lay-tech}, but \TeX{} is never pronounced \emph{tecks}. It is typeset using the \lstinline|\LaTeX{}| macro or with the capitalisation ``LaTeX''.
\subsection{Language Structures}
There are two major language structures that we encounter when using \LaTeX{}; \textit{macros} and \textit{environments}.
\subsubsection{Macros}
Macros (or commands) tell \LaTeX{} how to do things. They use the following syntax
\begin{lstlisting}
\commandname
% or
\commandname[optional args]{required args}
\end{lstlisting}
Macros provide functionality for layouts, symbols, styles, etc. As shown below, common font styles can be invoked using macros.
\begin{description}
    \item[Bold face] \lstinline|\textbf{Text}| --- \textbf{Text}
    \item[Italics] \lstinline|\textit{Text}| --- \textit{Text}
    \item[Emphasis] \lstinline|\emph{Text}| --- \emph{Text} (either upright or italics depending on surrounding text)
    \item[Underline] \lstinline|\underline{Text}| --- \underline{Text}
\end{description}
We can also define custom macros that combine other macros or simplify repetitive instructions.
\begin{lstlisting}
\newcommand{\commandname}[number_of_arguments]{command_body}
\end{lstlisting}
Arguments can be referenced inside the command body with the \lstinline|#argument_number| syntax.
For example
\begin{lstlisting}
\newcommand{\boldanditalics}[1]{\textbf{\textit{#1}}}
\boldanditalics{Bold and italics text}
\end{lstlisting}
\newcommand{\boldanditalics}[1]{\textbf{\textit{#1}}}
\boldanditalics{Bold and italics text}
\subsubsection{Environments}
Environments are used to format large blocks of text which often contain many lines or multiple macros. Environments use opening \lstinline|\begin| and closing \lstinline|\end| tags so that everything inside those tags will be formatted in a special manner depending on the type of the environment.
\begin{lstlisting}
\begin{environment_name}[optional_arguments]{required_arguments}
    ...
\end{environment_name} 
\end{lstlisting}
Common environments include \lstinline{figure}, \lstinline{equation}, \lstinline{itemize} (these will be discussed later), etc.
\newpage
\section{Basic Structure}
\subsection{Sections}
Sections are used to divide the document into parts. A new section is started with the \lstinline|\section{section_name}| macro. Section titles are formatted to be bold and larger than regular text. The number preceding the titles are automatically determined.

The table of contents (\lstinline{\tableofcontents}) is also generated from these section macros, so that page numbers and section numbers are set automatically.
\subsection{Subsections}
We can split sections into smaller subsections \lstinline|\subsection{Subsections}|,
\subsubsection{Subsubsections}
and also subsubsections \lstinline|\subsubsection{Subsubsections}|.
\subsection*{Unnumbered sections}
\addcontentsline{toc}{subsection}{Unumbered sections}
We can remove section numbering by using the starred version of the section macro
e.g.\ \lstinline|\subsection*{Unnumbered sections}|.

As this also removes the section from the table of contents, we can manually add it using
\begin{lstlisting}
\addcontentsline{toc}{subsection}{Unumbered sections}
\end{lstlisting}
remembering to place this immediately after the section macro so that the reference is set to the
correct location.
\subsection{Lists}
Unordered (bullet) lists are produced by the \lstinline{itemize} environment,
where each list entry starts by using the \lstinline{\item} command, which
also generates the bullet symbol.
\begin{lstlisting}
\begin{itemize}
    \item List entries ...
    \item We can ...
          \begin{itemize}
              \item We can ...
                    \begin{itemize}
                        \item The marker ...
                              \begin{itemize}
                                  \item List markers, ...
                              \end{itemize}
                    \end{itemize}
          \end{itemize}
\end{itemize}
\end{lstlisting}
\begin{itemize}
    \item List entries start with the \lstinline{\item} macro and are indicated by the black dot
    \item We can create multiple entries
          \begin{itemize}
              \item We can nest lists by creating another \lstinline{itemize} environment
                    \begin{itemize}
                        \item The marker changes in each nested list to reflect the depth
                              \begin{itemize}
                                  \item List markers, spacing, and other behaviour can be customised with the \lstinline{enumitem} package
                              \end{itemize}
                    \end{itemize}
          \end{itemize}
\end{itemize}
Numbered (ordered) lists use the same syntax as unordered lists but use the \lstinline{enumerate} environment.
\begin{lstlisting}
\begin{enumerate}
    \item List entries ...
    \item Nested lists ...
          \begin{enumerate}
              \item But use ...
                    \begin{enumerate}
                        \item Such as ...
                    \end{enumerate}
          \end{enumerate}
\end{enumerate}
\end{lstlisting}
\begin{enumerate}
    \item List entries are numbered automatically
    \item Nested lists are also numbered
          \begin{enumerate}
              \item But use a different number format
                    \begin{enumerate}
                        \item Such as lowercase letters and roman numerals
                    \end{enumerate}
          \end{enumerate}
\end{enumerate}
We can change the top-level number format by specifying a value to the \lstinline{label}
parameter.
\begin{lstlisting}
\begin{enumerate}[label=label_specifier]
    \item ...
\end{enumerate}
\end{lstlisting}
The following label specifiers can be used to override the default numbering format:
\begin{enumerate}[label=\arabic*.]
    \item --- \lstinline|[label=\arabic*.]| (Default)
\end{enumerate}
\begin{enumerate}[label=<\Roman*>]
    \item --- \lstinline|[label=<\Roman*>]|
\end{enumerate}
\begin{enumerate}[label=\roman*]
    \item --- \lstinline|[label=\roman*]|
\end{enumerate}
\begin{enumerate}[label=(\alph*)]
    \item --- \lstinline|[label=(\alph*)]|
\end{enumerate}
\begin{enumerate}[label=Part \Alph*:]
    \item --- \lstinline|[label=Part \Alph*:]|
\end{enumerate}
\newpage
\section{Mathematics}
One of \LaTeX{}'s strengths is how it formats mathematical expressions.
There are two ways to format mathematical expressions; inline using \lstinline|\( \)| and display style using \lstinline|\[ \]|.

Mathematical expressions can be contained ``inline'' (within) text and require less space:
\begin{lstlisting}
Let \(\mathbb{N}\) denote the set of all natural numbers.
\end{lstlisting}
Let \(\mathbb{N}\) denote the set of all natural numbers.

Mathematical expressions typeset outside paragraph text appear as standalone, display style math:
\begin{lstlisting}
\[
    \lim_{\Delta{t} \to \infty} \frac{f\left( t + \Delta{t} \right) - f\left( t \right)}{\Delta{t}}
\]
\end{lstlisting}
\[
    \lim_{\Delta{t} \to \infty} \frac{f\left( t + \Delta{t} \right) - f\left( t \right)}{\Delta{t}}
\]
Note that we commonly use \lstinline|\equation| environments for automatic vertical spacing and equation numbering
(as with section headings).
\begin{lstlisting}
\begin{equation}
    a^2 + b^2 = c^2
\end{equation}
\end{lstlisting}
\begin{equation}
    a^2 + b^2 = c^2
\end{equation}
We can use the starred version of this environment to remove the equation label.
\begin{lstlisting}
\begin{equation*}
    a^2 + b^2 = c^2
\end{equation*}
\end{lstlisting}
\begin{equation*}
    a^2 + b^2 = c^2
\end{equation*}
\subsection{Paired Delimiters}
\begin{table}[H]
    \centering
    \begingroup
    \renewcommand{\arraystretch}{1.2}
    \begin{tabular}{c c c c}
        \toprule
        \textbf{Name}  & \textbf{\LaTeX{} Command}                & \textbf{Inline}                  & \textbf{Display Style}                         \\
        \midrule
        Parentheses    & \lstinline|\left( a \right)|             & \(\left( a \right)\)             & \(\displaystyle \left( a \right)\)             \\ % chktex 37
        Brackets       & \lstinline|\left[ a \right]|             & \(\left[ a \right]\)             & \(\displaystyle \left[ a \right]\)             \\
        Braces         & \lstinline|\left\{ a \right\}|           & \(\left\{ a \right\}\)           & \(\displaystyle \left\{ a \right\}\)           \\
        Angle brackets & \lstinline|\left\langle a \right\rangle| & \(\left\langle a \right\rangle\) & \(\displaystyle \left\langle a \right\rangle\) \\
        Pipes          & \lstinline|\left\lvert a \right\lvert|   & \(\left\lvert a \right\lvert\)   & \(\displaystyle \left\lvert a \right\lvert\)   \\
        Double Pipes   & \lstinline|\left\lVert a \right\lVert|   & \(\left\lVert a \right\lVert\)   & \(\displaystyle \left\lVert a \right\lVert\)   \\
        Ceiling        & \lstinline|\left\lceil a \right\rceil|   & \(\left\lceil a \right\rceil\)   & \(\displaystyle \left\lceil a \right\rceil\)   \\
        Floor          & \lstinline|\left\lfloor a \right\rfloor| & \(\left\lfloor a \right\rfloor\) & \(\displaystyle \left\lfloor a \right\rfloor\) \\
        \bottomrule
    \end{tabular}
    \endgroup
    \caption{Paired Delimiters in \LaTeX{}.} % \label{}
\end{table}
Note that we can declare custom paired delimiters for the final five examples using the following syntax:
\begin{lstlisting}
\DeclarePairedDelimiter{\paired_delimiter_name}{left_delimiter}{right_delimiter}
\end{lstlisting}
Here are a few suggestions
\begin{lstlisting}
\DeclarePairedDelimiter{\ceil}{\lceil}{\rceil}
\DeclarePairedDelimiter{\floor}{\lfloor}{\rfloor}
\DeclarePairedDelimiter{\abracket}{\langle}{\rangle}
\DeclarePairedDelimiter{\abs}{\lvert}{\rvert}
\DeclarePairedDelimiter{\norm}{\lVert}{\rVert}
\end{lstlisting}
\DeclarePairedDelimiter{\ceil}{\lceil}{\rceil}
\DeclarePairedDelimiter{\floor}{\lfloor}{\rfloor}
\DeclarePairedDelimiter{\abracket}{\langle}{\rangle}
\DeclarePairedDelimiter{\abs}{\lvert}{\rvert}
\DeclarePairedDelimiter{\norm}{\lVert}{\rVert}
\subsection{Arithmetic Operators}
\begin{table}[H]
    \centering
    \begingroup
    \renewcommand{\arraystretch}{1.2}
    \begin{tabular}{c c c c}
        \toprule
        \textbf{Name}  & \textbf{\LaTeX{} Command}                     & \textbf{Inline}                       & \textbf{Display Style}                              \\
        \midrule
        Addition       & \lstinline|a + b|                             & \(a + b\)                             & \(\displaystyle a + b\)                             \\
        Subtraction    & \lstinline|a - b|                             & \(a - b\)                             & \(\displaystyle a - b\)                             \\ % chktex 8
        Multiplication & \lstinline|a \cdot \left( b \times c \right)| & \(a \cdot \left( b \times c \right)\) & \(\displaystyle a \cdot \left( b \times c \right)\) \\ % chktex 37
        Inequalities   & \lstinline|a \ll b < c \leq d|                & \(a \ll b < c \leq d\)                & \(\displaystyle a \ll b < c \leq d\)                \\
        Fractions      & \lstinline|\frac{a}{b}|                       & \(\frac{a}{b}\)                       & \(\displaystyle \frac{a}{b}\)                       \\
        Superscripts   & \lstinline|a^2|                               & \(a^2\)                               & \(\displaystyle a^2\)                               \\
        Subscripts     & \lstinline|a_i|                               & \(a_i\)                               & \(\displaystyle a_i\)                               \\
        Square root    & \lstinline|\sqrt{a}|                          & \(\sqrt{a}\)                          & \(\displaystyle \sqrt{a}\)                          \\
        \bottomrule
    \end{tabular}
    \endgroup
    \caption{Arithmetic operators in \LaTeX{}.} % \label{}
\end{table}
\subsection{Common Large Operators}
\begin{table}[H]
    \centering
    \begingroup
    \renewcommand{\arraystretch}{2.2}
    \begin{tabular}{c c c c}
        \toprule
        \textbf{Name} & \textbf{\LaTeX{} Command}                      & \textbf{Inline}                        & \textbf{Display Style}                               \\
        \midrule
        Summations    & \lstinline|\sum_{i=1}^n i|                     & \(\sum_{i=1}^n i\)                     & \(\displaystyle \sum_{i=1}^n i\)                     \\
        Limits        & \lstinline|\lim_{x \to 0} \frac{\sin{x}}{x}|   & \(\lim_{x \to 0} \frac{\sin{x}}{x}\)   & \(\displaystyle \lim_{x \to 0} \frac{\sin{x}}{x}\)   \\
        Derivatives   & \lstinline|\odv{f}{x} \pdv{}{t} \odif{\Omega}| & \(\odv{f}{x} \pdv{}{t} \odif{\Omega}\) & \(\displaystyle \odv{f}{x} \pdv{}{t} \odif{\Omega}\) \\
        Integrals     & \lstinline|\int_0^\infty e^{-x^2} \odif{x}|    & \(\int_0^\infty e^{-x^2} \odif{x}\)    & \(\displaystyle \int_0^\infty e^{-x^2} \odif{x}\)    \\
        Union         & \lstinline|\bigcup_{i=1}^n S_i|                & \(\bigcup_{i=1}^n S_i\)                & \(\displaystyle \bigcup_{i=1}^n S_i\)                \\
        \bottomrule
    \end{tabular}
    \endgroup
    \caption{Common large operators in \LaTeX{}.} % \label{}
\end{table}
\subsection{Common Mathematical Functions}
\begin{table}[H]
    \centering
    \begingroup
    \renewcommand{\arraystretch}{1.2}
    \begin{tabular}{c c c c}
        \toprule
        \textbf{Name}     & \textbf{\LaTeX{} Command}             & \textbf{Inline}               & \textbf{Display Style}                      \\
        \midrule
        Sine              & \lstinline|\sin{\left( x \right)}|    & \(\sin{\left( x \right)}\)    & \(\displaystyle \sin{\left( x \right)}\)    \\ % chktex 37
        Inverse Sine      & \lstinline|\arcsin{\left( x \right)}| & \(\arcsin{\left( x \right)}\) & \(\displaystyle \arcsin{\left( x \right)}\) \\ % chktex 37
        Logarithm         & \lstinline|\log{\left( x \right)}|    & \(\log{\left( x \right)}\)    & \(\displaystyle \log{\left( x \right)}\)    \\ % chktex 37
        Natural Logarithm & \lstinline|\ln{\left( x \right)}|     & \(\ln{\left( x \right)}\)     & \(\displaystyle \ln{\left( x \right)}\)     \\ % chktex 37
        Exponential       & \lstinline|\exp{\left( x \right)}|    & \(\exp{\left( x \right)}\)    & \(\displaystyle \exp{\left( x \right)}\)    \\ % chktex 37
        \bottomrule
    \end{tabular}
    \endgroup
    \caption{Common large operators in \LaTeX{}.} % \label{}
\end{table}
\subsection{Multi-line Equations}
As \lstinline{equation} only allows single line equations, we can use other environments to group multiple equations into one environment.
\subsubsection{Gather}
The \lstinline{gather} environment allows us to display a set of consecutive equations with multiple lines.
New lines are separated using \lstinline|\\|.
\begin{lstlisting}
\begin{gather}
    \sum_{i = 0}^n f\left( i \right) = 
        f\left( 0 \right) + f\left( 1 \right) 
        + \cdots + f\left( n \right) \\
    \prod_{i = 0}^n f\left( i \right) = 
        f\left( 0 \right) \times f\left( 1 \right) 
        \times \cdots \times f\left( n \right)
\end{gather}
\end{lstlisting}
\begin{gather}
    \sum_{i = 0}^n f\left( i \right) = f\left( 0 \right) + f\left( 1 \right) + \cdots + f\left( n \right) \\
    \prod_{i = 0}^n f\left( i \right) = f\left( 0 \right) \times f\left( 1 \right) \times \cdots \times f\left( n \right)
\end{gather}
\newpage
\subsubsection{Align}
The \lstinline{align} environment allows us to display consecutive equations that are also aligned. The alignment is determined by the placement of the \lstinline{&} character. This alignment character breaks the equation into ``columns'' that are either right or left aligned, following the pattern: \lstinline{rlrlrl}\dots.
\begin{lstlisting}
\begin{align}
%   R &  L  & R & L & R & L &  R  & L
    R & = L & R & = & = & L & R = & L
\end{align}
\end{lstlisting}
\begin{align}
    R & = L & R & = & = & L & R = & L
\end{align}
This can be illustrated using a table.
\begin{table}[H]
    \centering
    \begin{tabular}{wc{4.2em}@{\makebox[\widthof{\&}][c]{|}}wc{4.2em}@{\makebox[\widthof{\&}][c]{|}}wc{4.2em}@{\makebox[\widthof{\&}][c]{|}}wc{4.2em}@{\makebox[\widthof{\&}][c]{|}}wc{4.2em}@{\makebox[\widthof{\&}][c]{|}}wc{4.2em}@{\makebox[\widthof{\&}][c]{|}}wc{4.2em}@{\makebox[\widthof{\&}][c]{|}}wc{4.2em}} % chktex 44
        Right & Left & Right & Left & Right & Left & Right & Left \\
        \midrule
    \end{tabular}
    \begin{tabular}{wr{4.2em}@{\&}wl{4.2em}@{\&}wr{4.2em}@{\&}wl{4.2em}@{\&}wr{4.2em}@{\&}wl{4.2em}@{\&}wr{4.2em}@{\&}wl{4.2em}}
        \(R\) & \(= L\) & \(R\) & \(=\) & \(=\) & \(L\) & \(R =\) & \(L\) \\
    \end{tabular}
    % \caption{} % \label{}
\end{table}
With this in mind, we can create aligned equations as shown below.
\begin{align}
    ax^2 + bx + c                                                    & = 0                                  \\
    a\left( x^2 + \frac{b}{a}x + \frac{c}{a} \right)                 & = 0                                  \\
    x^2 + \frac{b}{a}x + \left( \frac{b}{2a} \right)^2 + \frac{c}{a} & = \left( \frac{b}{2a} \right)^2      \\
    \left( x + \frac{b}{2a} \right)^2                                & = \frac{b^2}{4a^2} - \frac{c}{a}     \\
    x + \frac{b}{2a}                                                 & = \frac{\pm \sqrt{b^2 - 4ac}}{2a}    \\
    x                                                                & = \frac{-b \pm \sqrt{b^2 - 4ac}}{2a}
\end{align}
\begin{align*}
    \symbf{v}_1 & = \symbf{w}_1                                                                                                   & \symbf{q}_1 & = \frac{\symbf{v}_1}{\norm{\symbf{v}_1}} \\
    \symbf{v}_2 & = \symbf{w}_2 - \proj_{\symbf{q}_1} \left( \symbf{w}_2 \right)                                                  & \symbf{q}_2 & = \frac{\symbf{v}_2}{\norm{\symbf{v}_2}} \\
    \symbf{v}_3 & = \symbf{w}_3 - \proj_{\symbf{q}_1} \left( \symbf{w}_3 \right) - \proj_{\symbf{q}_2} \left( \symbf{w}_3 \right) & \symbf{q}_3 & = \frac{\symbf{v}_3}{\norm{\symbf{v}_3}} \\
                & \vdots                                                                                                          &             & \vdots                                   \\
    \symbf{v}_i & = \symbf{w}_i - \sum_{j = 1}^{i - 1} \proj_{\symbf{q}_j} \left( \symbf{w}_i \right)                             & \symbf{q}_i & = \frac{\symbf{v}_i}{\norm{\symbf{v}_i}}
\end{align*}
If we want to write normal text in math mode, we need to use the \lstinline{\text} macro.
\begin{lstlisting}
\begin{align*}
    \text{Text in text mode} \\
    Text in math mode
\end{align*}
\end{lstlisting}
\begin{align*}
    \text{Text in text mode} \\
    Text in math mode
\end{align*}
Notice that spaces are ignored in math mode.
\subsection{Horizontal Spacing}
If we want to add horizontal space we can use the following macros:
\begin{lstlisting}
\begin{align*}
    A & \!     B \\
    A &        B \\
    A & \,     B \\
    A & \:     B \\
    A & \;     B \\
    A & \      B \\
    A & \quad  B \\
    A & \qquad B
\end{align*}
\end{lstlisting}
\begin{align*}
    A & \!     B \\
    A & B        \\
    A & \,     B \\
    A & \:     B \\
    A & \;     B \\
    A & \      B \\
    A & \quad  B \\
    A & \qquad B
\end{align*}
\subsection{Additional Symbols}
\LaTeX{} provides lots of symbols that can be installed from the \href{https://ctan.org}{Comprehensive TeX Archive Network} that are used in math mode, including the greek alphabet (shown below). A short (but extensive) list can be found at \href{https://www.rpi.edu/dept/arc/training/latex/LaTeX_symbols.pdf}{The Great, Big List of \LaTeX{} Symbols}.
\begin{equation*}
    \alpha \beta \gamma \delta \epsilon \varepsilon \zeta \eta \theta \vartheta \iota \kappa \lambda \mu \nu \xi \pi \varpi \rho \varrho \sigma \varsigma \tau \upsilon \phi \varphi \chi \psi \omega
\end{equation*}
\begin{equation*}
    \Gamma \Delta \Theta \Lambda \Xi \Pi \Sigma \Upsilon \Phi \Psi \Omega
\end{equation*}
Bringing all of these together can give pretty equations like:
\begin{equation*}
    \Gamma\left( z \right) = \frac{e^{-\gamma z}}{z} \prod_{k = 1}^\infty \left( 1 + \frac{z}{k} \right)^{-1} e^{\frac{z}{k}}.
\end{equation*}
\begin{equation*}
    \symbf{G}_{\mu \nu} + \symbf{\Lambda} \symbf{g}_{\mu \nu} = \kappa \symbf{T}_{\mu \nu}
\end{equation*}
\begin{equation*}
    \ce{Hg^2+ ->[I-] HgI2 ->[I-] [Hg^{II}I4]^2-}
\end{equation*}
\begin{equation*}
    i \hbar \pdv{}{t} \Psi\left( x,\: t \right) = -\frac{\hbar^2}{2m} \pdv[order=2]{}{x} \Psi\left( x,\: t \right) + V\left( x,\: t \right) \Psi\left( x,\: t \right)
\end{equation*}
\newpage
\section{Figures, Tables, and Code}
\LaTeX{} allows us to use figures and tables which can be added raw or by using floats.
We generally use floats to allow \LaTeX{} to algorithmically place figures on a page, and
move to the next page if it encounters a vertical overflow.

The float environment for figures is \lstinline{figure} and \lstinline{table} for tables.
Floats are containers for objects that cannot be displayed over multiple pages.
They should always have a descriptive caption (\lstinline{\caption}) so that the reader does not have to rely on the text, and also so that we can reference them using hyperlinks.

If we wish to give the object we want to reference a marker, we can use the \lstinline{label} macro.
\begin{lstlisting}
\caption{This is a caption for the figure. The figure numbering is automatic.}\label{fig:cat}
\end{lstlisting}
see the source code for the cat figure for this implementation.

Note that labels or reference markers should always be placed immediately after the object we want to reference.
This ensures that we maintian correct page references.

We can reference figures using the following syntax:
\begin{lstlisting}
Figure~\ref{fig:cat} shows a cat, and Figure~\ref{fig:turtle} shows a turtle.
\end{lstlisting}
Figure~\ref{fig:cat} shows a cat, and Figure~\ref{fig:turtle} shows a turtle.
\begin{figure}[H]
    \centering
    \includegraphics[width=0.5\textwidth]{figures/cat.jpg}
    \caption{This is a caption for the figure. The figure numbering is automatic.}\label{fig:cat}
\end{figure}
\begin{figure}[H]
    \centering
    \includegraphics[width=1cm,height=5cm]{figures/cat.jpg}
    \caption{Providing a second dimension may skew the image.}
\end{figure}
\begin{figure}[H]
    \centering
    \begin{subfigure}{0.47\textwidth}
        \centering
        \includegraphics[height=3.5cm]{figures/rabbit.jpg}
        \caption{The first subfigure.}
    \end{subfigure}
    \begin{subfigure}{0.47\textwidth}
        \centering
        \includegraphics[height=3.5cm]{figures/turtle.jpg}
        \caption{The second subfigure.}\label{fig:turtle}
    \end{subfigure}
    \caption{A caption for both subfigures.}
\end{figure}
\begin{table}[H]
    \centering
    \caption{An example of a table. Captions are placed above tables.}
    \begin{tabular}{|l|c r|}
        \hline
              & Column 1 & Column 2 \\
        \hline
        Row 1 & 7        & 2        \\
        Row 2 & 8        & 9        \\
        Row 3 & 2        & 0        \\
        Row 4 & 4        & 1        \\
        \hline
    \end{tabular}
\end{table}

Lists of figures and tables can be printed similarly to a table of contents with \lstinline{\listoffigures} and  \lstinline{\listoftables}

Source code can also be included with the \lstinline{listing} environment. In addition to the environment, code can also be included with the \lstinline{\lstinline} macro (this is what I have been doing for all the macros in this document)

\begin{figure}[H]
    \caption{An example listing}
    \begin{lstlisting}[language=c++,keywordstyle=\color{blue}]
#include <iostream>

int main() {
    std::cout << "Hello World!" << std::endl;
    return 0;
}\end{lstlisting}
\end{figure}

\subsection{References \& Labels}
Throughout this document you may have noticed that everything is numbered (e.g. equations, sections, figures, tables). It is incredibly easy to refer to these things with the label and reference system in \LaTeX{}. Everything that is numbered (and some things that are not) can have a label attached with the \lstinline|\label{labelname}| macro. The number can then be later referred to with the \lstinline|\ref{labelname}| macro. This means that if you go back and add a figure, all your reference to later figures will be automatically updated to reflect the new figure names.

Even better, when the \lstinline{hyperref} package is included (by \lstinline|\usepackage{hyperref}| in the preamble), all of these references are turned into hyperlinks to the item. This means that you can click on a reference and go straight to the related equation, figure, or table. This package also turns all the entries in the table of contents and table of figures into a link, to allow for easy navigation of the document

If you want to reference the page that a label is on, you can do that with the \lstinline|\pageref{labelname}| macro.
Below are some examples of references.
\begin{itemize}
    \item Reference to a figure -- Figure \ref{fig:cat}
    \item Reference to a subfigure -- Figure \ref{fig:turtle}
    \item Reference to table -- Table \ref{tab:placement-specs}
    \item Reference to a section (with pageref) -- Section \ref{sec:lists} on page \pageref{sec:lists}
\end{itemize}

\subsection{\LaTeX{} (usually) knows better than you}
A mistake many people new to \LaTeX{} make is trying to force images to go in particular places. This is often very hard and not necessary. \LaTeX{} is very good at placing floats in places that look good, and do not break the flow or layout of the document too much.

There are several placement specifiers that we can provide to the float commands in order to give \LaTeX{} hints on where we want the float to go. These go right after begin float macro. e.g. \lstinline|\begin{figure}[placement specifier] ... \end{figure}|. Table~\ref{tab:placement-specs} shows all of the available placement specifiers.

\begin{table}
    \centering
    \caption{Placement specifiers for floats. Multiple of these can be specified}
    \label{tab:placement-specs}
    \begin{tabular}{ll}
        \toprule
        Spec. & Location                                                 \\
        \midrule
        h     & Place \emph{approximately} here                          \\
        t     & Place at top of page                                     \\
        b     & Place at bottom of page                                  \\
        p     & Place on page for only floats                            \\
        !     & Override internal placement parameters (force placement) \\
        \bottomrule
    \end{tabular}
\end{table}

\section{Citations}
There are many ways to manage bibliographies, citations, and reference lists in \LaTeX{}, and many of them are outdated or have been superseded by newer alternatives. The method that I use is with a combination of BibLaTeX and Biber. BibLaTeX is the frontend in \LaTeX{} that handles citations and printing the bibliography. Biber is the backend which manages the database of all the references. The \lstinline{biblatex} package needs to be included with \lstinline|\usepackage[style=ieee]{biblatex}| in the preamble. The IEEE style can be replaced with others, such as APA. This changes both the citation and bibliography style.

References are stored in a database file with a \lstinline{.bib} extension. An example database file for the sources used in this document is shown below. Each entry in the database file refers to an source, with the necessary fields filled. In the preamble, the database file needs to be added with \lstinline|\addbibresource{references.bibF}|.

These sources can be cited with \lstinline|\cite{referencename}| (no parentheses) or \lstinline|\parencite{referencename}| (with parentheses). For example \parencite{colu92} \cite{phil99}

At the end of the document, the bibliography can be printed with \lstinline{\printbibliography}. It will only print sources that are actually cited in the document.

\section{Other Use Cases}

\subsection{In-Place Diagrams}

In cases where it may be convenient/elegant to draw simple plots/diagrams with LaTeX itself, the \lstinline{tikz} packages provides support for this. In addition, the plots shown in Figures~\ref{frequencyplot} and \ref{graphplot} also require the \lstinline{pgfplots} package.

% \begin{figure}
%     \centering
%     \begin{tikzpicture}
%         \begin{axis}[xtick = {-6200, 6200}, xticklabels = {\tiny-6200, \tiny6200}, scaled x ticks = false, ytick = {0,1}, yticklabels = {0, \(a\)}, ymin = -0.5, ymax = 1.5, xlabel = Frequency (Hz), ylabel = Magnitude]
%             \addplot[color = black, <-] coordinates {(-12000,0) (-6200,0)};
%             \addplot[color = black, <-] coordinates {(12000,0) (6200,0)};
%             \addplot[color = black, dotted] coordinates {(-6200,0) (-6200,1)};
%             \addplot[color = black, dotted] coordinates {(6200,0) (6200,1)};
%             \addplot[color = black] coordinates {(-6200,1) (6200,1)};
%         \end{axis}
%     \end{tikzpicture}
%     \caption{Frequency Plot from EGB242}
%     \label{frequencyplot}
% \end{figure}

% \begin{figure}
%     \centering
%     \begin{tikzpicture}
%         \begin{axis}[legend style={legend pos=north west}]
%             \legend{\(x=-\sqrt{y}\),\(x=\sqrt{y}\),\(x=y-2\)}
%             \addplot [
%                 red,
%                 domain=-2:0,
%                 samples=200,
%             ]{x^2};

%             \addplot [
%                 blue,
%                 domain=0:3,
%                 samples=200,
%             ]{x^2};

%             \addplot [
%                 green,
%                 domain=-3:4,
%                 samples=200,
%             ]{x+2};

%             \addplot [
%                 black,
%                 dashed,
%                 domain=-2:2,
%                 samples=200,
%                 pin edge = solid
%             ]{1} node [pos=0.95,pin=355:{\(\{y=1\}\)}] {};

%             \addplot [
%                 black,
%                 dashed,
%                 domain=1:3,
%                 samples=200,
%                 pin edge = solid
%             ]{4} node [pos=0.95,pin=355:{\(\{y=4\}\)}] {};

%             \addplot [
%                 black,
%                 dashed,
%                 domain=-0.5:0.5,
%                 samples=200,
%                 pin edge = solid
%             ]{0} node [pos=0.95,pin=355:{\(\{y=0\}\)}] {};

%             \addplot[
%                 only marks,
%             ]
%             coordinates{(-1,1) (2,4)}
%             node [pos=0, pin=270:{\([-1,1]\)}] {}
%             node [pos=1, pin=135:{\([2,4]\)}] {};

%             \addplot [
%                 only marks,
%             ]
%             coordinates{(0,0)}
%             node [pos=0, pin=270:{\([0,0]\)}] {};
%         \end{axis}
%     \end{tikzpicture}
%     \caption{Graph Plot from MXB105}
%     \label{graphplot}
% \end{figure}

\subsection{Circuit Diagrams}

It may be useful at some point to be able to construct circuit diagrams in LaTeX, and so the \lstinline{circuitikz} package can be used in order to implement this functionality. Shown below in Figure \ref{circuitdiagram} is an example of a circuit diagram one might construct.

% \begin{figure}
%     \centering
%     \begin{circuitikz}
%         \draw (0,0)
%         to[american current source=0.42A] (0,4)
%         to[short] (2,4)
%         to[R= 350 <\ohm>, v^>=\(\)] (2,0)
%         to[short] (0,0)
%         (1,2) node[scale=2]{\(\circlearrowright\)}
%         (1,2) node[scale=0.66]{\(i_s\)};

%         \draw (2,4)
%         to[R= 2.97 <\ohm>, v^>=\(\)] (6,4)
%         to[R= 16.5 <\ohm>, v_>=\(\)] (6,2)
%         to[empty led, v_>=26.2 <\volt>] (6,0)
%         to[R= 2.97 <\ohm>, v_>=\(\)] (2,0)
%         (4,2) node[scale=2]{\(\circlearrowright\)}
%         (4,2) node[scale=0.66]{\(i_1\)};
%         \draw (6,4)
%         to[R=0.27 <\ohm>, v^>=\(\)] (10,4)
%         to[R= 16.5 <\ohm>, v_>=\(\)] (10,2)
%         to[empty led, v_>=26.2 <\volt>] (10,0)
%         to[R=0.27 <\ohm>, v_>=\(\)] (6,0)
%         (8,2) node[scale=2]{\(\circlearrowright\)}
%         (8,2) node[scale=0.66]{\(i_2\)};
%         \draw (10,4)
%         to[R=0.36 <\ohm>, v^>=\(\)] (14,4)
%         to[R= 16.5 <\ohm>, v_>=\(\)] (14,2)
%         to[empty led, v_>=26.2 <\volt>] (14,0)
%         to[R=0.36 <\ohm>, v_>=\(\)] (10,0)
%         (12,2) node[scale=2]{\(\circlearrowright\)}
%         (12,2) node[scale=0.66]{\(i_3\)};
%     \end{circuitikz}
%     \caption{Circuit Diagram from EGB120}
%     \label{circuitdiagram}
% \end{figure}

\subsection{Matrices/Vectors}

It may also be convenient at some point to be able to express matrices and vectors in MATLAB. These have their own environments which come from the \lstinline{amsmath} package, and can be used to produce the following:

\begin{align*}
    \begin{bmatrix}
        x'_1(t) \\
        x'_2(t) \\
        x'_3(t) \\
        x'_4(t)
    \end{bmatrix}
     & =
    \begin{bmatrix}
        -0.1 & 0.05 & 0    & 0    \\
        0.05 & -0.2 & 0.05 & 0.05 \\
        0.05 & 0.05 & -0.2 & 0.05 \\
        0    & 0    & 0.05 & -0.1
    \end{bmatrix}
    \begin{bmatrix}
        x_1(t) \\
        x_2(t) \\
        x_3(t) \\
        x_4(t)
    \end{bmatrix}
    +
    \begin{bmatrix}
        500 \\
        250 \\
        100 \\
        10
    \end{bmatrix}
\end{align*}

\newpage
\section{Other Resources}
\begin{itemize}
    \item The \LaTeX{} Wikibook -- Good reference documentation and guides -- \url{https://en.wikibooks.org/wiki/LaTeX}
    \item The \TeX{} Stack Exchange -- Answers to most of your questions -- \url{https://tex.stackexchange.com/}
    \item Comprehensive \TeX{} Archive Network (CTAN) -- Package database -- \url{https://ctan.org/}
\end{itemize}

\newpage
\printbibliography

\end{document}